%%%%%%%%%%%%%%%%%%%%%%%%%%%%%%%%%%%%%%%%%
% Two Column One Page Curriculum Vitae
% LaTeX Template
% Version 1.1 (24/1/13)
%
% This template has been downloaded from:
% http://www.LaTeXTemplates.com
%
% Original author:
% Alessandro (The CV Inn)
%
% Modified for use for:
% Emil Gedda
% 
% IMPORTANT: THIS TEMPLATE NEEDS TO BE COMPILED WITH XeLaTeX
%
% This template uses several fonts not included with Windows/Linux by
% default. If you get compilation errors saying a font is missing, find the line
% on which the font is used and either change it to a font included with your
% operating system or comment the line out to use the default font.
% 
%%%%%%%%%%%%%%%%%%%%%%%%%%%%%%%%%%%%%%%%%

%-------------------------------------------------------------------------------
%	PACKAGES AND OTHER DOCUMENT CONFIGURATIONS
%-------------------------------------------------------------------------------

\documentclass[10pt]{article} % Font size - 10pt, 11pt or 12pt

\usepackage[hmargin=1.25cm, vmargin=2cm]{geometry} % Document margins

\usepackage{marvosym} % Required for symbols in the colored box
\usepackage{ifsym} % Required for symbols in the colored box

\usepackage[usenames,dvipsnames]{xcolor} % Allows the definition of hex colors

% Fonts and tweaks for XeLaTeX
\usepackage{xltxtra,xunicode}
\usepackage{fontspec}
\usepackage{fontawesome} % Website icons
\defaultfontfeatures{Mapping=tex-text}
%\setromanfont{Font Awesome 5 Free} % Main document font
%\setsansfont{Times New Roman} % Font for your name at the top
%\setmonofont[Scale=MatchLowercase]{Andale Mono}

% Colors for links, text and headings
\usepackage{hyperref}
\definecolor{linkcolor}{HTML}{104b87} % Blue-gray color for links
\definecolor{shade}{HTML}{F5DD9D} % Peach color for the contact information box
\definecolor{text1}{HTML}{2b2b2b} % Main document font color, off-black
\definecolor{headings}{HTML}{701112} % Dark red color for headings
% Other color palettes: shade=B9D7D9 and linkcolor=A40000; shade=D4D7FE and linkcolor=FF0080

\hypersetup{colorlinks,breaklinks, urlcolor=linkcolor, linkcolor=linkcolor} % Set up links and colors

\usepackage{fancyhdr}
\pagestyle{fancy}
\fancyhf{}
\cfoot{\thepage}

% Headers and footers can be added with the \lhead{} \rhead{} \lfoot{} \rfoot{} commands
% Example footer:
%\rfoot{\color{headings} {\sffamily Last update: \today}. Typeset with Xe\LaTeX}

\renewcommand{\headrulewidth}{0pt} % Get rid of the default rule in the header

\usepackage{titlesec} % Allows creating custom \section's

% Format of the section titles
\titleformat{\section}{\color{headings}
\scshape\Large\raggedright}{}{0em}{}[\color{black}\titlerule]

\titlespacing{\section}{0pt}{0pt}{5pt} % Spacing around titles

\begin{document}

\color{text1} % Sets the default text color for the whole document to the color defined as 'text1'

%-------------------------------------------------------------------------------
%	TITLE
%-------------------------------------------------------------------------------

\par{\centering{\sffamily\Huge Emil Gedda}\\[1cm] % Your name
%{\color{headings}\fontspec[Variant = 2]{Zapfino} Curriculum {Vit\fontspec[Variant = 3]{Zapfino}\ae}\\[15pt]\par} % Curriculum vitae text in the Zapfino font
	
%-------------------------------------------------------------------------------

\begin{minipage}[t]{0.5\textwidth} % Start the left-hand side of the page
	
%-------------------------------------------------------------------------------
%	WORK EXPERIENCE
%-------------------------------------------------------------------------------

\section{Work Experience} 

{\raggedleft\textsc{Summer 2019}\par}

{\raggedright\large Software Engineering Intern\\ \textit{EF Education First}\\[5pt]}

    \normalsize{Software Engineering Intern at EF Class in Chelsea, London. Working with microservices in Go orchestrated by Kubernetes on AWS developing free educational software for schools.}\\

{\raggedleft\textsc{Jun 2018 -- Jun 2019}\par}

{\raggedright\large Software Developer\\ \textit{SAAB Group}\\[5pt]}

    \normalsize{Software Developer at SAAB Group, Electronic Warfare department. Worked with the Linux kernel as well as writing high performant C++98/14 software with modern tools such as Jenkins, Docker, Google test, and LLVM.}\\

%------------------------------------------------
% WORK EXPERIENCE 1
%------------------------------------------------
{\raggedleft\textsc{Aug 2016 -- Aug 2018}\par}

{\raggedright\large Teaching assistant\\ \textit{KTH Royal Institute of Technology}\\[5pt]}

    \normalsize{Teaching assistant in \textsc{DD1361: Programming Paradigms}, an introductory course into different programming paradigms, such as functional programming in Haskell, and logic programming in Prolog. }\\

    \normalsize{Teaching assistant in \textsc{DD1388: Program System Construction Using C++}. The course aims to teach students C++98/11/14 and how to apply the standard library while writing efficient, testable code.}\\

%------------------------------------------------
% WORK EXPERIENCE 3
%------------------------------------------------

{\raggedleft\textsc{Summer 2010}\par}

{\raggedright\large Software Developer\\ \textit{OpenRatio}\\[5pt]}

\normalsize{Summer intern and developer of their Windows Phone platform, writing pure C\# for their enterprise mobile platform. Their platform enabled clients to develop cross platform apps and manage said apps without any programming knowledge, using a platform-independent web interface.}\\


\end{minipage} % End the left-hand side of the page
\hfill
\begin{minipage}[t]{0.44\textwidth} % Start the right-hand side of the page

%-------------------------------------------------------------------------------
%	COLORED BOX
%-------------------------------------------------------------------------------

\colorbox{shade}{\textcolor{text1}{
\begin{tabular}{c|p{7cm}}
\raisebox{-4pt}{\textifsymbol{18}} & Ernst Ahlgrens väg 4, 11255 Stockholm \\ % Address
\raisebox{-3pt}{\Mobilefone} & +46707505910 \\ % Phone number
\raisebox{-1pt}{\Letter} &
    \href{mailto:emil.gedda@emilgedda.se}{emil.gedda@emilgedda.se} \\
    \faLinkedin &
    \href{https://www.linkedin.com/EmilGedda}{https://linkedin.com/EmilGedda}
    \\
    \faGithub &
    \href{https://www.github.com/EmilGedda}{https://github.com/EmilGedda}
    \\
\end{tabular}
}
}\\[10pt]

%-------------------------------------------------------------------------------
%	EDUCATION
%-------------------------------------------------------------------------------

\section{Education} 

\begin{tabular}{ll} % Start a table with two columns, one for dates and one for qualifications

%------------------------------------------------
% Master of Science
%------------------------------------------------

2018 -- \textsc{Present} & \textbf{Master of Science} \\ 
& \textsc{Computer Science} \\ 
& \textit{KTH Royal Institute of Technology}\\
&\\
	 
2017 -- 2018 & \textbf{Master of Science} \\ 
& \textsc{Computer Science} \\ 
& \small Exchange studies in USA\\
& \textit{University of Illinois at} \\
& \textit{Urbana-Champaign} \\
&\\
%------------------------------------------------
% Bachelor of Science
%------------------------------------------------

2014 -- 2017 & \textbf{Bachelor of Science} \\ 
& \textsc{Computer Science} \\ 
& \textit{KTH Royal Institute of Technology}\\
&\\
	
%-------------------------------------------------------------------------------

\end{tabular}\\[10pt]

%-------------------------------------------------------------------------------
%	COMPUTER SKILLS
%-------------------------------------------------------------------------------

\section{Skills} 

\begin{tabular}{ll}
Intermediate
& \textsc{python}, x86\_64 ASM, SQL,\\
    & git, \textsc{C\#}, Bash, Golang, Cyber Security \\[4pt]
Advanced
& C18, C++20, GNU/Linux, \\
    & Haskell, \textsc{java}, Prolog \\[4pt]
\end{tabular}

%-------------------------------------------------------------------------------
%	COMMUNICATION SKILLS
%-------------------------------------------------------------------------------

\section{Interests} 

My main areas of interest are performance optimization, functional programming, and cyber security.
I have a burning passion for high performant modern C++ code and operating systems. 
When it comes to cyber security my main interests are reverse engineering and penetration testing.

%-------------------------------------------------------------------------------
\end{minipage} % End right-hand side of the page
\clearpage

\begin{minipage}[t]{0.5\textwidth} % Start the left-hand side of the page
\section{Projects}

{\raggedleft\textsc{Spring 2016}\par}

{\raggedright\large kOS x86\_64 OS\\ \textit{Current spare time project}  \\ \small URL:
    \href{https://github.com/EmilGedda/kOS}{https://github.com/EmilGedda/kOS} \\[5pt]}

    \normalsize{kOS is a 64bit operating system for the x86 platform. kOS is written in C++2a using libc++, libc++abi, and libunwind. kOS
        is still very much in its planning stages. kOS will feature a microkernel designed for fast IPC and modularity, while still being simple enough for educational purposes.}\\

{\raggedleft\textsc{Summer and Fall 2016}\par}

{\raggedright\large Hattis\\ \textit{Spare time project}\\ \small URL: \href{https://www.github.com/EmilGedda/hattis}{https://github.com/EmilGedda/hattis}\\[5pt]}


\normalsize{Hattis is a simple command line interface tool for the online
    programming problem judge, kattis. Hattis allows users to submit solutions
    and track the submission progress live, all from the command line. The tool is
    written in Haskell, exercising modern techniques regarding error
    handling and correctness.}\\

\end{minipage} % End the left-hand side of the page
\hfill
\begin{minipage}[t]{0.44\textwidth} % Start the right-hand side of the page
%-------------------------------------------------------------------------------	
\section{Papers}

{\raggedleft\textsc{Spring 2017}\par}

{\raggedright\large Analysis of The Precision Time Protocol under different
    forms of system load\\ \textit{Bachelor Thesis} \\[2pt] \small URN:
    \href{http://www.diva-portal.org/smash/record.jsf?pid=diva2:1106630}{urn:nbn:se:kth:diva-208493}\\[5pt]}

    \normalsize{Evaluted the most popular implementation of
    the Precision Time Protocol, \textsc{IEEE1588}, under different types of system
    load. This involved setting up a local network of devices and synchronizing
    their system clocks while stressing different subsystems of the connected
    devices. The results showed that the accuracy and precision of PTP suffers
    greatly when a client is under heavy load.}\\


	
\end{minipage} % End right-hand side of the page

\end{document}  
